\documentclass[12pt]{article}
\usepackage{times}
\usepackage{alltt}

\begin{document}

\title{Machine extraction}
\author{Simon Thompson}
\date{G\"{o}teborg, August 2011}
\maketitle

\section{Overview}

\begin{itemize}
\item Formalise process of traces to QSM state machine
\item Formalise processes from QSM state machine to QuickCheck FSM.
\end{itemize}


\section{Traces to QSM state machine}

\subsection{Basic QSM}

Definition of concrete traces:
\begin{verbatim}
item  ::= {module, function, arguments, result}
trace ::= {pos, item*} | {neg, item*}
\end{verbatim}
If $T$ is a set of traces then \texttt{t} is a positive trace if \texttt{\{pos,t\}}
\ensuremath{\in T}; similarly for negative traces.

An abstraction function is a function 
\begin{alltt}
\ensuremath{\alpha} :: \{module, function, arguments, result\} -> \ensuremath{\cal{A}}
\end{alltt}
for some type $\cal{A}$.. The \emph{abstract traces} are given by \texttt{\{pos,map(abs,t)\}} and \texttt{\{neg,map(abs,t)\}}.

A set $T$ of traces is \emph{consistent} iff for all \texttt{\{neg,t\}} \ensuremath{\in T} and for all \texttt{\{pos,t'\}}\ensuremath{\in T}, \texttt{t} is not an initial segment of \texttt{t'}. 

The QSM algorithm builds a deterministic state machine from a consistent set of traces. The construction fails if applied to an inconsistent set. The  machine built, ${\cal{M}}_T = <{\cal{S}}, s_0, \cal{F}, \cal{T}>$, has the form 
\begin{itemize}
\item
$\cal{S}$ is a non-empty finite set of states,
\item
$s_0 \in \cal{S}$ is the initial state of the system,
\item
$\cal{F} \subseteq \cal{S}$ is a set of failing states, and 
\item
$\cal{T}$ is a set of transitions of the form $(s_1,a,s_2)$ where $s_i \in \cal{S}$ and $a \in \cal{A}$.
\end{itemize}
For states $s_1$ and $s_{k+1}$ and a trace \texttt{t} = $a_1{}a_2{}\ldots{}a_k$ we write $s_1 \stackrel{\texttt{t}}{\rightarrow} s_{k+1}$ if for each $i=1\ldots{}k$ there is some $a_i$ so that $(s_i,a_i,s_{i+1}) \in \cal{T}$. 

We say that the machine ${\cal{M}}_T$ \emph{accepts} \texttt{t} if $s_0 \stackrel{\texttt{t}}{\rightarrow} s$ with $s \not \in \cal{F}$; the machine \emph{rejects} \texttt{t} if $s_0 \stackrel{\texttt{t}}{\rightarrow} s$ with $s \in \cal{F}$, or there is no path through the machine ${\cal{M}}_T$ labelled by \texttt{t}. Since the machine is deterministic, each trace will either be accepted or rejected by the machine, but not both.


The crucial property of the machine ${\cal{M}}_T$ is that 
for all positive traces \texttt{t} in $T$, ${\cal{M}}_T$ accepts \texttt{t}, and
for all negative traces \texttt{t} in $T$, ${\cal{M}}_T$ rejects \texttt{t}. This is true by construction.


\subsection{Abstractions and QSM}

We have written ${\cal{M}}_T$ for the machine inferred from a set of traces $T$, which might be concrete or might arise from an abstraction of a concrete set. In this section we'll work with a fixed set of concrete traces, and use $T_\alpha$ for the set of traces after abstraction and ${\cal{M}}_\alpha$ for the machine generated from $T_\alpha$.

It would be nice to relate machines produced for related abstractions, e.g.\ in the case where an abstraction is a composition of two others, with $ \alpha = \beta \circ \gamma$. However, even in the case of relating machines for concrete and abstract sets of traces, ${\cal M}_{id}$ and ${\cal M}_\alpha$, this is not possible in general since the machines generated by the QSM algorithm are not canonical: depending on the order in which states are handled by the algorithm, different machines result.

\subsection{SM: state machines}

A \emph{state machine} of type $\cal D$ has the form ${\cal{M}} = \;<{\cal{S}}, s_0, {\cal T}, d_0, \eta, \phi, \psi>$
\begin{itemize}
\item
$\cal{S}$ is a non-empty finite set of states,
\item
$s_0 \in \cal{S}$ is the initial state of the system,
\item
$\cal{T}$ is a set of transitions of the form $(s_1,a,s_2)$ where $s_i \in \cal{S}$ and $a \in \cal{A}$.
\item
$d_0 \in \cal D$ is the initial value of the state data,
\item
$\eta$ is a partial function
\[
\eta :: \{{\cal S},{\cal A},{\cal D}\} \rightarrow \cal D
\]
taking a state, an action and the value of the state data before the transition to the value after,
\item
$\phi, \psi$ are partial functions
\[
\phi, \psi :: \{{\cal S},{\cal A},{\cal D}\} \rightarrow \cal B
\]
where $\cal B$ is the Boolean type. These are the pre- and post-conditions of the transition from the given state, action and value of the state data.

\end{itemize}






\end{document}
